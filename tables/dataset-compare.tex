\begin{table}[h]
  \begin{center}
  \begin{tabular}{l l l l r}
  Dataset & \#Samples & \rotatebox[origin=lB]{90}{Meshes} & \rotatebox[origin=lB]{90}{Manifold} & \rotatebox[origin=lB]{90}{Balanced annot.}\\
  \midrule
  \rowcolor{green!20}
  \multicolumn{1}{l|}{DONUT}  & 30,000   & \cm & \cm & \cm \\
  \multicolumn{1}{l|}{MANTRA \cite{mantra}} & 43,100 & -- & \cm & -- \\
  \multicolumn{1}{l|}{ABC \cite{abc}} & 1,000,000+ & \cm & --   & -- \\
  \multicolumn{1}{l|}{Thingi10K \cite{thingi}} & 10,000 & \cm & -- & -- \\
  \rowcolor{green!20}
  \multicolumn{1}{l|}{EuLearn  \cite{eulearn}} & 3,300 & \cm & \cm & \cm \\
  \midrule
  \end{tabular}
  \end{center}
  \vspace{-2mm}
  \caption{\textbf{Overview of existing datasets and their capabilities.} We summarize here the main characteristics of existing datasets with topological annotations. Besides EuLearn, all existing datasets with topological annotations come with downsides, discussed in Section~\ref{ssec:existing_datasets}. However, since EuLearn seems to be the most promising dataset, we carried out an extensive analysis to (1) highlight limitations that make it unreliable for further experiments and (2) motivate the use of DONUT (see Appendix~\ref{ssec:suppl_eulearn_analysis}). \textit{Note:} The number of samples for MANTRA only takes into account 2-manifolds.}
  \label{tab:datasets}
\end{table}